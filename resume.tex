\documentclass[12pt]{article}

\usepackage[utf8]{inputenc}
\usepackage{amsmath,amsfonts,amssymb}

\title{Méthodes formelles}
\author{Daniel Sanz\\
   Université de Fribourg,\\
   \texttt{daniel.sanz@unifr.ch}}
\date{\today}




\begin{document}
\maketitle

\begin{abstract}
    Ceci est un résumé non officiel du cours de méthodes 
    formelles du professeur Ultes Nietzche. Il s'agit
    principalement de ces slides traduites en français
    ainsi que quelques exos en guise d'exemple.
\end{abstract}

\section*{Introduction}
Les formules logiques, les prédicats entre autres peuvent être 
utilisés afin d'éxprimer de l'information sur l'état d'un programme.
\(x = 10;\) indique que x doit impérativement avoir la valeur
10.
\section*{Pré \& post-condition}
\paragraph*{Définition}
Une précondition P nous indique ce qui peut être considéré comme 
vrai avant même l'execution d'une séquence d'instructions S.\\
Une postcondition Q nous indique ce qui sera vrai après l'execution
des instructions S.
\paragraph*{Notation}
On écrit : \(\{P\}\; S\; \{Q\}\) qui veut dire que si P est vrai alors,
après l'execution de S, Q est vrai. Il s'agit d'un triplet d'Hoarce.\\

\paragraph*{Exemple}
\begin{equation*}
    \{x = 2;\}\; x = x \cdot 3;\; \{x = 6;\}
\end{equation*}

On utilise \(\hat{x}\) comme notation de la variable \(x\) pour indiquer
la valeur de \(x\) après l'execution du programme et \(x\) avant l'execution.\\
\paragraph*{Exemple}

\begin{align*}  
    \{true\}\; x &= x+1;\; \{\hat{x}>x\}\\
    \{true\}\; x &= x+1;\; \{\hat{x} = x + 1\}
\end{align*}

\section*{Les assertions}
Il est possible d'écrire des prédicats entre deux lignes de code. On
présume alors que ce prédicats est la postcondition de la ligne de code
précédente et qu'il est la précondition de la ligne suivante.

\paragraph*{Définition}

De tels prédicats sont dits \emph{assertions} ou \emph{annotations}.
Pour savoir si des triplets sont corrects il faut tout d'abord transformer
le programme S en une formule \(\phi_{S}\). Ainsi il est possible de prouver
l'exactitude d'un triplet:

\begin{equation*}
    \{P\}\; S\; \{Q\}
\end{equation*}

en verifiant la formule: 

\begin{equation*}
    P \wedge \phi_{S} \rightarrow Q
\end{equation*}

ou de façon analogue:

\begin{equation*}
    \phi_{S} \rightarrow (P \rightarrow Q)
\end{equation*}

\paragraph*{Exemple}
\begin{align*}
    \{true\}\; x &= 10; \; \{x > 0\}\\
    \phi_{S} &\equiv x = 10
\end{align*}
donc, $(true \wedge ( x = 10))\rightarrow (x>10)$.
\paragraph*{Exemple}
\begin{align*}
    \{x \neq 0\}\; x &= 1/x;\\
    x &= 1/x; \; \{\hat{x} = x\}
\end{align*}
Soit $x'' = 1/x$ et $\hat{x} = 1/x''$ alors on vérifie:
\begin{equation*}
    (x \neq 0) \wedge (x'' = 1/x) \wedge (\hat{x} = 1/x'') \rightarrow \hat{x} = x
\end{equation*}
Ce qui est vrai par du calcul élémentaire. Biensûr, ici on ne tient pas
compte de la précision limitée de \emph{floats}.
\end{document}