\documentclass[12pt]{article}

\usepackage[utf8]{inputenc}
\usepackage{amsmath,amsfonts,amssymb}
\usepackage{algorithm}
\usepackage{algorithmic}

\title{Méthodes formelles}
\author{Daniel Sanz\\
   Université de Fribourg,\\
   \texttt{daniel.sanz@unifr.ch}}
\date{\today}




\begin{document}
\maketitle

\begin{abstract}
    Ceci est un résumé non officiel du cours de méthodes 
    formelles du professeur Ultes Nietzche. Il s'agit
    principalement de ces slides traduites en français
    ainsi que quelques exos en guise d'exemple.
\end{abstract}

\section*{Introduction}
Les formules logiques, les prédicats entre autres peuvent être 
utilisés afin d'éxprimer de l'information sur l'état d'un programme.
\(x = 10;\) indique que x doit impérativement avoir la valeur
10.
\section*{Pré \& post-condition}
\paragraph*{Définition}
Une précondition P nous indique ce qui peut être considéré comme 
vrai avant même l'execution d'une séquence d'instructions S.\\
Une postcondition Q nous indique ce qui sera vrai après l'execution
des instructions S.
\paragraph*{Notation}
On écrit : \(\{P\}\; S\; \{Q\}\) qui veut dire que si P est vrai alors,
après l'execution de S, Q est vrai. Il s'agit d'un triplet d'Hoarce.\\

\paragraph*{Exemple}
\begin{equation*}
    \{x = 2;\}\; x = x \cdot 3;\; \{x = 6;\}
\end{equation*}

On utilise \(\hat{x}\) comme notation de la variable \(x\) pour indiquer
la valeur de \(x\) après l'execution du programme et \(x\) avant l'execution.\\
\paragraph*{Exemple}

\begin{align*}  
    \{true\}\; x &= x+1;\; \{\hat{x}>x\}\\
    \{true\}\; x &= x+1;\; \{\hat{x} = x + 1\}
\end{align*}

\section*{Les assertions}
Il est possible d'écrire des prédicats entre deux lignes de code. On
présume alors que ce prédicats est la postcondition de la ligne de code
précédente et qu'il est la précondition de la ligne suivante.

\paragraph*{Définition}

De tels prédicats sont dits \emph{assertions} ou \emph{annotations}.
Pour savoir si des triplets sont corrects il faut tout d'abord transformer
le programme S en une formule \(\phi_{S}\). Ainsi il est possible de prouver
l'exactitude d'un triplet:

\begin{equation*}
    \{P\}\; S\; \{Q\}
\end{equation*}

en verifiant la formule: 

\begin{equation*}
    P \wedge \phi_{S} \rightarrow Q
\end{equation*}

ou de façon analogue:

\begin{equation*}
    \phi_{S} \rightarrow (P \rightarrow Q)
\end{equation*}

\paragraph*{Exemple}
\begin{align*}
    \{true\}\; x &= 10; \; \{x > 0\}\\
    \phi_{S} &\equiv x = 10
\end{align*}
donc, $(true \wedge ( x = 10))\rightarrow (x>10)$.
\paragraph*{Exemple}
\begin{align*}
    \{x \neq 0\}\; x &= 1/x;\\
    x &= 1/x; \; \{\hat{x} = x\}
\end{align*}
Soit $x'' = 1/x$ et $\hat{x} = 1/x''$ alors on vérifie:
\begin{equation*}
    (x \neq 0) \wedge (x'' = 1/x) \wedge (\hat{x} = 1/x'') \rightarrow \hat{x} = x
\end{equation*}
Ce qui est vrai par du calcul élémentaire. Biensûr, ici on ne tient pas
compte de la précision limitée des \emph{floats}.

\section*{Les clauses If}
Soit la clause générale \emph{If} suivante. Avec la précondition P et la 
postcondition Q.

\begin{align*}
    \{P\}\; &\textnormal{if}(condition)\;\{\textnormal{progI}\}\\
            &\textnormal{else} \; \{\textnormal{progII}\}\\
            &\qquad \qquad \qquad \;\{Q\}
\end{align*}
Afin de prouver cette clause il faut procéder aux transformation
suivantes (et les prouver indépendament):
\begin{align*}
    \left\{P \wedge condition \right\}\; &\textnormal{progI}\; \left\{Q\right\} \qquad \text{\underline{et}}\\
    \left\{P \wedge \lnot condition \right\}\; &\textnormal{progII}\; \left\{Q\right\}
\end{align*}
Les deux triplets doivent être vrais pour vérifier la clause.

\paragraph*{Exemple}
Soit la clause \emph{If} suivant avec sa précondition P et postcondition Q
respéctive.
\begin{align*}
    \overbrace{\left\{true\right\}}^\text{P}\; &\textnormal{if} (x<0) \; \left\{x = -x\right\}\\
                          &\textnormal{else} \; \left\{x = x\right\}\\
                          &\qquad \qquad \quad \underbrace{\left\{\hat{x}\geqslant 0\right\}}_{\text{Q}}
\end{align*}
Ainsi on fait la transformation :
\begin{align*}
    \left\{true \wedge (x<0)\right\}\; &x = -x \left\{\hat{x}\geqslant 0\right\} \quad \text{\underline{et}}\\
    \left\{true \wedge \lnot (x<0)\right\}\; &x = x \left\{\hat{x}\geqslant 0\right\}
\end{align*}
Formellement, il faudrait encore faire las transformation du programme en 
une formule logique $\phi_{S}$ afin de prouver les deux triplets. Mais
il est évident que c'est juste.

\section*{Les boucles}
On s'intérresse ici à comment prouver les boucles. En particulier la boucle
while. Mais tout d'abord nous avons besoin de quelques définitions et
d'outils supplémentaires.
\paragraph*{Définitions}
\begin{itemize}
    \item Quand une boucle se termine et que le résultat éspéré est 
    atteint on dit qu'il s'agit d'une \emph{exactitude partielle}.
    \item Quand il est garantit qu'une boucle se atteint une fin on 
    dit : \emph{termination}
    \item Quand les deux conditions précédentes sont vérifiées on parle
    d'\emph{exactitude total}
\end{itemize}

\subsection*{Invariant de boucle}
Il s'agit d'une formule logique qui est vraie dans les cas suivants:
\begin{itemize}
    \item avant la boucle
    \item avant chaque execution du corps de la boucle
    \item après chaque execution du corps de la boucle
    \item après la boucle
\end{itemize}
et elle doit rendre la postcondition vraie. Donc grosso merdo c'est vrai
tout le temps, d'où l'invariance.

\subsection*{Variant de boucle}
Il s'agit d'une expression évaluée dans les entiers $\mathbb{N}^{+}$ qui
\begin{itemize}
    \item est décrémentée de 1 à chaque itération
    \item ne peut pas aller en dessous de 0
\end{itemize}
Donc en d'autres termes int i = CST; et dans la boucle i = i - 1;

\section*{Boucle While}
Soit la boucle suivante:
\begin{align*}
    \left\{P\right\}\; &initialisation;\\
    &\textnormal{while}(condition)\left\{\textnormal{loop body}\right\};\; \left\{Q\right\}
\end{align*}
On vérifie l'exactitude partielle et la terminaison séparement.

\paragraph*{L'exactitude partielle} à l'aide de l'invariant de boucle
nous donne les formules suivantes:
\begin{align*}
    \left\{P\right\}\; &initialisation; &\left\{Inv\right\}\\
    \left\{Inv \wedge condition\right\}\; &\textnormal{loop body}; &\left\{Inv\right\}\\
    \left\{Inv \wedge \lnot condition\right\}\; &\textnormal{skip}; &\left\{Q\right\}
\end{align*}

\paragraph*{La terminaison} nous donne encore la formule suivante à 
vérifier:
\begin{equation*}
    \left\{int var \wedge var > 0\right\}\; \textnormal{loop body};\; \left\{var > var'\geqslant 0\right\}
\end{equation*}

\paragraph*{Exemple} 
Soit ce programme qui calcul de façon itérative la somme des entiers de 1 à $n$.
\begin{align*}
    \overbrace{\left\{n>0 \wedge x = 1\right\}}^{P}\; &sum = 1; \textnormal{//initialisation}\\
    \textnormal{while}(x<n) \{\\
                            &x = x + 1;\\
                            &sum = sum + x;\\
                            &\}; \quad \qquad \qquad \underbrace{\left\{sum = n(n+1)/2\right\}}_{Q}
\end{align*}
\emph{Exactitude partielle}\\
On pose l'invariant \(\text{sum} = x(x+1)/2\).\\
On commence par verifier l'initialisation.
\begin{align*}
    \{n>0 \wedge x = 1\}\; \text{sum} = 1 \;\{\text{sum} = x(x+1)/2\}\\
    (n>0) \wedge (x=1) \wedge (\text{sum} = 1) \rightarrow (\text{sum} = x(x+1)/2)
\end{align*}
Ce qui est vrai car \(1 = 1(1+1)/2\). Puis nous vérifions le corps de la
boucle avec le triplet suivant:
\begin{align*}
    \{(\text{sum} = x(x+1)/2) \wedge (x<n)\}\; &x = x + 1;\\
    &\text{sum} = \text{sum} + x; \{\text{sum}\; x(x+1)/2\}
\end{align*}
Donc,
\begin{align*}
    &(\text{sum} = x(x+1)/2)\\
    \wedge\; &(x<n)\\
    \wedge\; &(x' = x + 1)\\
    \wedge\; &(\text{sum}' = sum + x') \rightarrow (\text{sum}' = x'(x'+1)/2)
\end{align*}

\begin{align*}
    \text{sum}' &= \frac{x(x+1)}{2} + x + 1 = \frac{x(x+1)}{2} + \frac{2x + 2}{2}\\
    &= \frac{x^{2} + x +2x + 2}{2} = \frac{x^{2} + 3x +2}{2}\\
    &= \frac{(x+2)(x+1)}{2} = \frac{x'(x'+ 1)}{2}
\end{align*}

Puis la dérnière formule à la fin de la boucle:
\[
  sum = \frac{x(x+1)}{2} \wedge x \geq n \to sum = \frac{n(n+1)}{2}
\]
Qui est vraie si \(x = n\) vu que les deux côtés de la formule sont les mêmes.
\emph{Terminaison}\\
Avec $n-x$ comme variant on obtient la formule:

\begin{align*}
   int\ var\ &\wedge \\
   var > 0\ &\wedge \\
   x = 1\ &\wedge \\
   n > 0\ &\wedge \\
   var = n - x\ &\wedge \\
   x' = x + 1\ &\wedge \\
   sum' = sum + x'\ &\wedge \\
  var' = n - x'\ &\to \\
  var > var' &\geq 0
\end{align*}
et donc
\[
  var' = n - x' = n - x + 1
\]

\[
  n - x > n - x - 1 \geq 0
\]
prouve la terminaison.

\section*{Notions sur les prédicats}
\subsection*{Faiblesses et forces des prédicats}
\emph{Définition :} $P$ est plus faible que $Q$ si et seulement si $Q \to P$\\
Ainsi $true$ est le plus faible des prédicats car il est impliqué par tout. Analoguement,
$false$ est le plus fort des prédicats car il implique tout.
\paragraph*{exemple}

\[\overbrace{-10 \leq x \leq 10}^Q \to \overbrace{-100 \leq x \leq 100}^{P}\]
En effet, $P$ est plus faible que $Q$ car $P$ décrit plus "d'état" que $Q$.
On peut comprendre ça comme quoi $Q$ est une implication plus stricte.

\subsection*{L'affaiblissement d'une précondition}
Si $P$ est plus faible que $P'$ c-à-d que $P' \to P$ alors prouver le triplet
\[\{P\}\;S\;\{Q\}\]
garantit la validité de $\{P'\} \; S \; \{Q\}$. En effet, 
\begin{align*}
    &(P \wedge \phi_{S}) \to Q\\
    &(P' \wedge \phi_{S}) \to (P \wedge \phi_{S}) \to Q
\end{align*}

\subsection*{Renforcement d'une postcondition}

Si $Q$ est plus fort que $Q'$ c-à-d que $Q \to Q'$ alors,
\begin{align*}
    \{P\}\;S\;\{Q\} \Rightarrow &(P \wedge \phi_{S}) \to (Q \to Q')\\
    &(P \wedge \phi_{S}) \to Q'\\
    &\{P\}\;S\;\{Q'\}
\end{align*}

\section*{Logique propositionelle}
\subsection*{Syntaxe de la logique propositionelle}
Une formule propositionelle logique peut se construire comme suit:
\begin{itemize}
    \item Les valeurs de vérité $\bot$ pour false et $\top$ pour true sont aussi bien des variables propositionelles que des formules.
    \item Si $F$ et $G$ sont des formules propo. logiques, alors
    \begin{itemize}
        \item $(F)$
        \item $\neg F$
        \item $F \wedge G$
        \item $F \vee G$
        \item $F \to G$
        \item $F \leftrightarrow G$
    \end{itemize}
    le sont aussi.
\end{itemize}
\emph{Définition: }Si $P$ est une variable, alors $P$ et $\neg P$ sont
dit litéraux.

\subsection*{Intérpretations des variables propositionnelles}

\emph{Définition: } Une intérpretation $I$ est une assignation de valeurs
de vérité à des variables propositionelles.

\paragraph*{exemple}
\begin{align*}
    I: \{ P \mapsto true, Q \mapsto false,\dots \} 
\end{align*}
\emph{Notation: }La valeur de vérité de la variable $p$ sous l'intéerpretation $I$
s'écrit $I[P]$ avec l'exemple d'avant ça donne:
\[
    I[P] = true, I[Q] = false
\]

\end{document}